\chapter{Student-Oriented Outcomes} \label{appendix2}

In addition to the theory presented in the previous chapters, a wide range of hands-on technical
content was required to be mastered in order to develop the research presented in this work.
Aspects which the author had limited or null experience with beforehand include:

\begin{itemize}
\item Development of Android applications.
\item ROS development (rosjava, installation from source, make files, etc.).
\item Use of Java, Python and C++ (over 700, 1500 and 800 lines written, respectively).
\item Use of \textit{dlib} library.
\item Development under Linux.
\item Use of github for version control, code re-view and collaboration.
\item Configuring of networks between devices.
\end{itemize}

Having undertaken the work in a research lab, there are also numerous practical non-technical skills which are transferable to future projects that have been built upon, including

\begin{itemize}
\item Demonstrations to laboratory visitors and others:

- Anara Sandygulova from University College Dublin on 14th April.

- Wafa Johal from Universit\'e de Grenoble on 5th May.

- Pooyan Fazli from Carnegie Mellon University on 16th April.

- Representatives of HKSTP, an R\&D company.

\item Conduction of user studies to adjust the direction of prototypes under development.
\item Presenting research to non-technical audiences such as:
		- Ergo therapist in Lausanne.
		- Ergo therapist in Geneva.
\item EPFL Open days.
\item CoWriter project meeting in IST Lisbon (Portugal).

\end{itemize}
