%%%%%%%%%%%%%%%%%%%%%%%%%%%%%%%%%%%%%%%%%%%%%%%%%%%%%%%%%%%%%%%%%%%%%%%%%%%
% This is a sample header for a sample dissertation. Fill in the name,
% and the other information. LaTeX will work out the table of
% content, the list of figures and of tables for you.
%%%%%%%%%%%%%%%%%%%%%%%%%%%%%%%%%%%%%%%%%%%%%%%%%%%%%%%%%%%%%%%%%%%%%%%%%%%

\newpage
\thispagestyle{empty}

% ******* Title page *******
% **************************

\vspace*{2cm}
\begin{center}
{\Large\bf Automatic behavioural adaptation\\
			based on engagement assessment in\\
			child-robot interaction\\} \vspace{2cm} {\large
Fernando Garcia\\
\vspace{2cm}
Computer Human Interaction Learning and Instruction \\
\'Ecole Polytechnique F\'ed\'erale de Lausanne}

\end{center}

\vspace{7cm}
\begin{center}
{\large A Thesis Submitted for the Degree of \\MSc Erasmus Mundus
in Vision and Robotics (VIBOT) \\\vspace{0.3cm} $\cdot$ 2015
$\cdot$}
\end{center}
\singlespacing


%ABSTRACT
\begin{abstract}
Sustaining long-term interventions in a human-robot interaction scenario with an engaged user requires a system able to provide a suitable response depending on the context of the specific situation. It becomes a must to capture the significant \textit{passive} information provided by the user during the interaction. This evaluation has to be reliable enough to provide a variety of behavioral responses.

This work takes place in the context of the CoWriter project, the first known robotic agent which can engage a user in the learning by teaching paradigm for handwriting. By leveraging simulated handwriting on a synchronized tablet display, a Nao humanoid robot has been configured as a suitably embodied handwriting partner.

Additionally, the particularities of the project allow further research in terms of artificial intelligence for decision making. For instance, the ability to provide behaviours based on the engagement level, or the generation of responses based on the quality assessment of the handwriting provided by the user.

Two in situ experiments were scheduled with a primary school classes to evaluate the human-robot interaction outcomes of the system and their improvements with respect to the original version. Finally, long term experiments are currently in progress with the collaboration of a professional centre and under supervision of an ergo-therapist. 

\vspace*{5cm}



\begin{center}
\begin{quote}
\it The more important reason is that the research itself provides an important long-run perspective on the issues that we face on a day-to-day basis.
\end{quote}
\end{center}
\hfill{\small Ben Bernanke}

\end{abstract}

\doublespacing

%\pagestyle{empty}
\pagenumbering{roman}
\setcounter{page}{1} \pagestyle{plain}


\tableofcontents

\listoffigures
\listoftables

\chapter*{Acknowledgments}
\addcontentsline{toc}{chapter}
         {\protect\numberline{Acknowledgments\hspace{-96pt}}}

I would like to express my gratitude to my supervisor S\'everin Lemaignan who gave me not only the opportunity to work on this project, but also his knowledge and valuable advice. I would also like to extend my appreciation to Alexis Jacq, our doctoral assistant, for his constant willingness to help in each single stage of this work. Special thanks also to Sharma Kshitij whose statistical knowledge helped me in the results analysis of this research and I came to know about so many new things I am really thankful to him. Finally, I would also like to thank Dina for her constant support, patience and courage.
\pagestyle{fancy}
