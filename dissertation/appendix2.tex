\chapter{Student-Oriented Outcomes} \label{apppendix2}

In addition to the theory presented in the previous chapters, a wide range of hands-on technical
content was required to be mastered in order to develop the system presented in this work.
Aspects which the author had limited experience with beforehand include

Development of Android applications.
ROS development (rosjava, installation from source, make files, etc.).
Use of Java and Python languages (over 600 and 1,700 lines written, respectively).
Development on Linux.
Use of github for version control, code re-view and collaboration, and issue tracking.
Configuring of networks between devices.

Having undertaken the work in a research lab, there are also numerous practical non-
technical skills which are transferable to future projects that have been built upon, including


• Demonstrations to laboratory visitors.

– Prof. Tetsuya Ogata from Laboratory for Intelligent Dynamics and Representation, Waseda University, Japan on 17th March
– Rune Fogh from Centre for Playware, Technical University of Denmark on 13th May
– Marcelo Worsley from Transformative Learning Technologies Lab, Stanford on 15th May
– Lukasz Kidziski from Universite Libre de Bruxelles on 20th May